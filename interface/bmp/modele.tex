\documentclass{article}
%\usepackage[greek, frenchb]{babel}
%\usepackage[latin1]{inputenc}%\usepackage[utf8]{inputenc}
%\usepackage[T1]{fontenc}
%
%\documentclass{article}
\usepackage[T1]{fontenc}
\usepackage[latin1]{inputenc}
\usepackage[greek, frenchb]{babel}
\usepackage{enumitem}
\usepackage{graphicx}
\usepackage{multirow}
\usepackage{multicol}
\usepackage{amsmath}
\usepackage{tabularx}
\usepackage{color}
%
%\usepackage{pslatex}
%\usepackage{chemist}
%\usepackage[version=3]{mhchem}
%\usepackage{cmbright}%\usepackage{ae, lmodern, arev}
%
%\usepackage[a4paper]{geometry}
\usepackage[a4paper, landscape]{geometry}
%
%%%%%%%%%%%%%%%%%%%%%%%%%%%%%%%%%%%%%%%%%%%%%%%%%%%%%%%%%%%%%%%
\geometry{top=1.3cm, bottom=1.3cm, left=1.5cm , right=1.5cm}
\setlength{\columnsep}{20pt}
%\setlength{\columnsep}{30pt}
%%%%%%%%%%%%%%%%%%%%%%%%%%%%%%%%%%%%%%%%%%%%%%%%%%%%%%%%%%%%%%%
%%%%%%%%%%%%%%%%%%%%%%%%%%%%%%%%%%%%%%%%%%%%%%%%%%%%%%%%%%%%%%%
\geometry{top=1.5cm, bottom=1.5cm, left=1.5cm , right=1.5cm}
%\geometry{top=1.3cm, bottom=1.3cm, left=1.3cm , right=1.3cm}
\setlength{\columnsep}{3cm}
%%%%%%%%%%%%%%%%%%%%%%%%%%%%%%%%%%%%%%%%%%%%%%%%%%%%%%%%%%%%%%%
%
%
%
\begin{document}
\pagestyle{empty}
%
%               POLICE
%
%\fontfamily{cmbright}\fontseries{m}\selectfont
\fontfamily{cmdh}\fontseries{m}\selectfont
\sffamily
%
%
%                 TABLEAU
%
\renewcommand{\arraystretch}{1.9}
\renewcommand{\arraystretch}{1.7}
\begin{tabular}{|l|l|l|l|l|l|l|l|l|}%\begin{center}%\multicolumn{4}{|c|}{}\\%\cline{2-7}\hline
\begin{center}
\begin{tabular}{rc|cl}
\multicolumn{4}{|c|}{}\\
\multicolumn{3}{|c|}{Cloison} & \multicolumn{3}{|c|}{Trou} & \multicolumn{3}{|c|}{Thermostat} \\
Sans & Perc� & d�mon & Max & Min & Nul & Sans & 1 & 2
\end{tabular}
\end{center}
%
%
%%%%%%%%%%%%%%%%%%%%%%%%%%%%%%%%%%%%%%%%%%%%%%%%
\end{document}
%%%%%%%%%%%%%%%%%%%%%%%%%%%%%%%%%%%%%%%%%%%%%%%%
%
%                 NUM�ROTATION
%
%\ding{32}
\begin{enumerate}[label=\alph*., leftmargin=0.3cm, itemsep=25pt]
%\begin{enumerate}[label=\arabic*., leftmargin=0.9cm, itemsep=3pt]
%\begin{enumerate}[label=\Roman*., leftmargin=0.1cm, itemsep=35pt]
\item \end{enumerate}
%
\begin{description}[leftmargin=1cm, labelindent=1cm, itemsep=3pt]\end{description}
%
%\ding{32}
\begin{itemize}[leftmargin=0.3cm, itemsep=1pt]
\item 
\item 
\end{itemize}
%                 IMAGE
%
\begin{center}
\includegraphics[width=.45\textwidth]{pression}
\end{center}
%
%                 TABLEAU
%
\renewcommand{\arraystretch}{1.9}
\renewcommand{\arraystretch}{1.7}
\begin{tabular}{r|l}%\begin{center}%\multicolumn{4}{|c|}{}\\%\cline{2-7}\hline
\begin{center}
\begin{tabular}{rc|cl}
\multicolumn{4}{|c|}{}\\
\end{tabular}
\end{center}
%
%                 COLONE
%
\begin{multicols}{2}\end{multicols}
\begin{multicols}{2}%\columnbreak\end{multicols}
%
%                 MINIPAGE
%
\begin{minipage}[c]{0.45\linewidth}
Contenu de ma premi�re colonne
\end{minipage}
\begin{minipage}[c]{0.45\linewidth}
Contenu de ma deuxi�me colonne
\end{minipage}
%
%                 COLONNE
%
\begin{columns}
\begin{column}{6cm}
Contenu de ma premi�re colonne
\end{column}
\begin{column}{6cm}
Contenu de ma deuxi�me colonne
\end{column}
\end{columns}
%
%
%                 BOITE
%
\begin{center}
\setlength{\fboxsep}{3mm}
\fbox{\parbox{17cm}{
%\fbox{\parbox{18cm}{
}}
\end{center}

\hspace{.3cm}
%
%
%%%%%%%%%%%%%%%%%%%%%%%%%%%%%%%%%%%%%%%%%%%%%%%%
\end{document}
%%%%%%%%%%%%%%%%%%%%%%%%%%%%%%%%%%%%%%%%%%%%%%%%
%PREAMBULE
\usepackage{pgfplots}
\usepackage{tikz}
\usepackage[european resistor, european voltage, european current]{circuitikz}
\usetikzlibrary{arrows,shapes,positioning}
\usetikzlibrary{decorations.markings,decorations.pathmorphing,
decorations.pathreplacing}
\usetikzlibrary{calc,patterns,shapes.geometric}
%FIN PREAMBULE
%
%    M�mo
%
%\fontfamily{cmbright}\fontseries{m}\selectfont \sffamily
%\Huge (g�ant)          \normalsize (normal)
%\huge (�norme)         \small (petit)
%\LARGE (tr�s grand)    \footnotesize (assez petit)
%\Large (plus grand)    \scriptsize (tr�s petit)
%\large (grand)         \tiny (minuscule)

%\rm roman              \sf sans serif
%\tt machine � �crire 	\bf gras
%\sl slanted            \it italique
%\sc petites majuscules

%
%    Ligne du tableau en gris
%
    \usepackage{xcolor}
    \makeatletter
    \def\colorhline#1{%
      \noalign{\ifnum0=`}\fi\begingroup\color{#1}\hrule \@height \arrayrulewidth\endgroup \futurelet
       \reserved@a\@xhline}
    \makeatother
%
%
\begin{enumerate}[label=\Roman*., leftmargin=0.1cm, itemsep=35pt]
%
%                                QUESTIONS PR�LIMINAIRE
%
\item {\large \bf Questions pr�liminaires}
  \begin{enumerate}[label=\arabic*., leftmargin=0.9cm, itemsep=3pt]
  \item 
  \end{enumerate}
%
%                                PROTOCOLE EXP�RIMENTAL
%
\item {\large \bf Protocole exp�rimental}
  \begin{enumerate}[label=\arabic*., leftmargin=0.9cm, itemsep=3pt]
  \item 
  \end{enumerate}
%
%                                EXPLOITATION DES R�SULTATS
%
\item {\large \bf Exploitation des r�sultats}

  \begin{enumerate}[label=\arabic*., leftmargin=0.9cm, itemsep=3pt]
  \item 
  \end{enumerate}
%
\end{enumerate}
%
%
%                                TABLEAU D'AVANCEMENT         D�BUT
%
{\large \item \bf Tableau d'avancement}
\begin{center}%%%%%%%%%%%%%%%%%%%%%%       
\renewcommand{\arraystretch}{1.5}
\begin{tabular}{|c|c|c|c|c|c|c|c|}
\cline{3-8}
\multicolumn{2}{r|}{} & \multicolumn{6}{c|}{ \hspace{3.3cm} $\to$ \hspace{6cm} }\\
\cline{3-8}
\multicolumn{2}{r|}{} & \hspace{2.3cm} & \hspace{2.3cm} & \hspace{2.3cm} & \hspace{2.3cm} & \hspace{0.9cm} & \hspace{0.9cm} \\
\hline
{\scriptsize \rm Avant la transformation } & 0 &  &  & 0 & 0 &  &  \\ \hline
{\scriptsize \rm Pendant la transformation } & x &  &  &  & &  &   \\ \hline
{\scriptsize \rm Apr�s la transformation } & x$_{\text{max}}$  &  &  &  & &  &   \\ \hline
\end{tabular}
\end{center}
%
%                                TABLEAU D'AVANCEMENT            FIN
%
%
%
%                                FIGURE ET MINIPAGE
%
%
\begin{figure}[htbp]
\begin{minipage}[c]{.45\linewidth}
\begin{center}
\includegraphics[scale=0.25]{./titre/heisenberg2}
%\caption{SiCF}
%\label{fig:image1}
\end{center}
\end{minipage}
\hfill
\begin{minipage}[c]{.45\linewidth}
\begin{center}
\includegraphics[scale=0.25]{./titre/heisenberg3}
%\caption{diagramme des t�ches.}
%\label{SiCP}
\end{center}
\end{minipage}
\end{figure}
%\textsc{\LARGE Compl�ment nutritionnel}\\[1.5cm]
~\\[1cm]

%%%%%%%%%%%%%%%%%%%%%
\section{Perspective et rep�re SiCP}
%%%%%%%%%%%%%%%%%%%%%
Cette section traite de la d�finition des coordonn�es intervenant dans la projection en perspective de SiCP
\subsection{Sch�ma}
%\includepdf{./illustration/repereSiCP.pdf}

\begin{figure}[h!]
	\begin{minipage}[c]{.46\linewidth}
	$\overrightarrow{r}  = \text{} .
	\begin{pmatrix}
		\cos \psi . \sin \phi \\
		\sin \psi . \sin \phi \\
		\cos \phi
	\end{pmatrix}$,

	$\overrightarrow{\psi} = \text{largeur} .
	\begin{pmatrix}
		- \sin \psi \\
		\cos \psi \\
		0
	\end{pmatrix}$,

	$\overrightarrow{\phi} = \text{hauteur} .
	\begin{pmatrix}
		- \cos \psi . \cos \phi \\
		- \sin \psi . \cos \phi \\
		\sin \phi
	\end{pmatrix}$.
	\end{minipage} \hfill
	\begin{minipage}[c]{.46\linewidth}
	\includegraphics[scale=.77]{./illustration/repereSiCP}
	\end{minipage}
\end{figure}

%
%
\begin{center}
 de SiCP
\end{center}


\subsection{Math�matique}
\begin{description}[leftmargin=0.1cm, itemsep=5pt]
\item{\bf System} : $\theta _\text{i}$.
\item{\bf Chaine} : {\bf r}$_\text{i}$.
\item{\bf Support} : {\bf R}$_\text{i}$.
\item{\bf Point de vue} : {\bf M}, {\bf i}$_\text{M}$,
                                   {\bf j}$_\text{M}$,
                                   {\bf k}$_\text{M}$.
\end{description}
\subsection{Classes}
\begin{description}[leftmargin=0.1cm, itemsep=5pt]
\item{\bf System} : nouveau[N].
\item{\bf Chaine} : chaine[N], support[12], largeur, hauteur.
\item{\bf Point de vue} : perspective, distance, psi, phi.
\end{description}
\subsection{Projection}
\begin{description}[leftmargin=0.1cm, itemsep=5pt]
\item{\bf System-Chaine} : r$_\text{i} = 
	\begin{pmatrix}
		\text{largeur} / 2 \text{N} ( \text{i} - \text{N} / 2 ) \\
		\text{hauteur}. \sin \theta _\text{i} \\
		\text{hauteur}. \cos \theta _\text{i}
	\end{pmatrix}$.
\item{\bf Point de vue} : 
	$\overrightarrow{r}  = \text{} .
	\begin{pmatrix}
		\cos \psi . \sin \phi \\
		\sin \psi . \sin \phi \\
		\cos \phi
	\end{pmatrix}$,
	$\overrightarrow{\psi} = \text{largeur} .
	\begin{pmatrix}
		- \sin \psi \\
		\cos \psi \\
		0
	\end{pmatrix}$,
	$\overrightarrow{\phi} = \text{hauteur} .
	\begin{pmatrix}
		- \cos \psi . \cos \phi \\
		- \sin \psi . \cos \phi \\
		\sin \phi
	\end{pmatrix}$.
\item{\bf Chaine-Rendu} : $\text{g}_\text{i} = 
	\begin{pmatrix}
	(\text{\bf r}_\text{i} - \text{\bf M}).{\text{\bf k}}_\text{M} + \text{hauteur} / 2 \\
	(\text{\bf r}_\text{i} - \text{\bf M}).{\text{\bf j}}_\text{M} + \text{largeur} / 2
	\end{pmatrix}$.
\end{description}
%%%%%%%%%%%%%%%%%%%%%%%%%%%%%%%%%%%%%%%%%%%%%%%%%%%%%%%%%%%%%%%%%%%%%%%%%%%%%%%%%%%%%
